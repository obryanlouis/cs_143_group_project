% !TEX root = Report.tex


\section{Mathematical Analysis}

* Test Case 2 can be solved analytically. Set up
the TCP equations that TCP Vegas / FAST-TCP are
engineered to solve, and then find (and justify!)
the equilibrium point for each phase of Test Case
2's operation. Compare your calculated results to
your experimental results; if they are wildly
different, discuss why this might be so.

In this section, we provide a mathematical analysis of TCP Vegas as applied to testcase 2 of our system. 

By example 9 of chapter 3 (3.9), the equilibrium solution to Vegas is the same as the optimal minimization of $\displaystyle\sum\limits_{i=0}^n \alpha_i \log(x_i)$ subject to the link capacity constraints $Rx <= c$. In the first time interval, there is only one flow, so this problem becomes

$$\frac{\alpha_1}{x_1}=\mu_1 + \mu_2 + \mu_3$$
$$\mu_1 (x_1 - 10) = 0$$
$$\mu_2 (x_1 - 10) = 0$$
$$\mu_3 (x_1 - 10) = 0$$

Solving these equations gives the solutions $x_1=10$Mbps, and $\mu_1 = 0.055, \mu_2 = 0, \mu_3 = 0$ (although these may be chosen arbitrarily to satisfy nonnegativity and the top equation). This rate matches the realized rate of flow 1 during simulation after the initial slow start.

The propogation delay $d_1$ is 100ms because each of the 5 links along the path from S1 to T1 has the same delay of 10ms. The base RTT is approximately 110ms because of the queueing delay. By Little's law, the equilibrium queueing delay is $\alpha /  x_1$, where in our simulation, $\alpha = 0.55 packets/ms$. This is equal to $\frac{0.55 packets/ms}{10 Mbps} * \frac{1024 B}{packet} * \frac{Mb}{10^6 b} * \frac{8 b}{B} * \frac{10^3 ms}{s} * (110 ms) = 50 ms$. So the equilibrium RTT is 150 ms. This calculation also agrees with the round trip time in our simulation. The corresponding window size is $150ms * 10Mbps = 183 packets$.

When flow 2 enters, its base RTT includes the queueing delay at link 1. The queueing delay is approximately 50 ms by the above, assuming that the queueing delay is the same as the previous equilibrium queueing delay. In addition, the propogation delay is approximately 66 ms. The KKT conditions to the minimization become

$$\frac{\alpha_1}{x_1}=\mu_1+\mu_2+\mu_3$$
$$\frac{\alpha_2}{x_2}=\mu_1$$
$$\mu_1 (x_1 + x_2 - 10) = 0$$
$$\mu_2 (x_1 - 10) = 0$$
$$\mu_3 (x_1 - 10) = 0$$

Solving these equations gives the solutions $x_1 = 5$Mbps, $x_2 = 5$ Mbps, and $\mu_1 = 0.110, \mu_2 = 0, \mu_3 = 0$. The simulation's flow rates of flow 1 and flow 2 converge toward this common value in the period before flow 3 is added, although they do not quite reach it before that time.

With the same base RTT for flow 1, the new equilibrium queueing delay is  $\frac{0.55 packets/ms}{5 Mbps} * \frac{1024 B}{packet} * \frac{Mb}{10^6 b} * \frac{8 b}{B} * \frac{10^3 ms}{s} * (110 ms) = 100 ms$. The new RTT of flow 1 is 200 ms, which agrees with the simulation's RTT. The equilibrium queueing delay for flow 2 is $\frac{0.55 packets/ms}{5 Mbps} * \frac{1024 B}{packet} * \frac{Mb}{10^6 b} * \frac{8 b}{B} * \frac{10^3 ms}{s} * (66ms + 50ms) = 105 ms$ since the base RTT is $66 ms + 50 ms$. So the equilibrium RTT for flow 2 is $105ms + 66ms = 171 ms$. This RTT does not match the simulation exactly, which produces round trip times of about 150 ms, but it is within a reasonable margin of error.  The corresponding window sizes for flows 1 and 2 are 122 packets and 104 packets. In turn, these measurements slightly differ from the simulation's values.

When flow 3 enters, its base RTT only depends on the propogation delay to T3 since the buffer at L3 is empty. So its base RTT is 66 ms, and its propogation delay is 60 ms. The KKT conditions become

$$\frac{\alpha_1}{x_1}=\mu_1+\mu_2+\mu_3$$
$$\frac{\alpha_2}{x_2}=\mu_1$$
$$\frac{\alpha_3}{x_3}=\mu_3$$
$$\mu_1 (x_1 + x_2 - 10) = 0$$
$$\mu_2 (x_1 - 10) = 0$$
$$\mu_3 (x_1 + x_3 - 10) = 0$$

The solution to these equations is $x_1 = 3.33$ Mbps, $x_2 = 6.67$ Mbps, $x_3 = 6.67$ Mbps, and $\mu_1 = .0825, \mu_2 = 0, \mu_3 = .0825$. These rates approximately agree with the simulation.

The queueing delays are

$$q_1 = \frac{0.55 packets/ms}{3.33 Mbps} \frac{1024 B}{packet} \frac{Mb}{10^6 b} \frac{8 b}{B} \frac{10^3 ms}{s} (110 ms) = 150 ms$$

$$q_2 = \frac{0.55 packets/ms}{6.67 Mbps} \frac{1024 B}{packet} \frac{Mb}{10^6 b} \frac{8 b}{B} \frac{10^3 ms}{s} (66 ms + 50 ms) = 78 ms$$

$$q_3 = \frac{0.55 packets/ms}{6.67 Mbps} \frac{1024 B}{packet} \frac{Mb}{10^6 b} \frac{8 b}{B} \frac{10^3 ms}{s} (66 ms) = 45 ms$$

and the corresponding window sizes are

$$w_1 = 102 packets$$
$$w_2 = 117 packets$$
$$w_3 = 90 packets $$

These correspond to RTT's of $T_1 = 250$ ms, $T_2 = 144$ ms, and $T_3 = 111$ ms. These results agree approximately with the simulation's results, however the simulation shows round trip times slightly lower. The windows sizes also approximately agree, however, they seem to converge to slightly lower results as well.

When flow 2 finishes, the new equilibrium equations are analogous to those given above for the case when only flow 1 and flow 2 are in the system; $x_1 = 5$ Mbps and $x_3 = 5$ Mbps, and $\mu_1 = 0, \mu_2 = 0, \mu_3 = .110$. The new queueing delays are 

$$q_1 = \frac{0.55 packets/ms}{5 Mbps} \frac{1024 B}{packet} \frac{Mb}{10^6 b} \frac{8 b}{B} \frac{10^3 ms}{s} (110 ms) = 100 ms$$

$$q_1 = \frac{0.55 packets/ms}{5 Mbps} \frac{1024 B}{packet} \frac{Mb}{10^6 b} \frac{8 b}{B} \frac{10^3 ms}{s} (66 ms) = 60 ms$$

and the window sizes are

$$w_1 = 122 packets$$
$$w_2 = 77 packets$$

The new RTT's are $T_1 = 200$ ms and $T_2 = 126$ ms. Again, these do not exactly match the simulation's results, however they are not too far away from the results, which were approximately 225 ms and 140 ms. The simulation window sizes were about 135 packets for flow 1 and 60 packets for flow 2.

Finally, when flow 3 is the only flow left, the KKT equations become

$$\frac{\alpha_3}{x_3}=\mu_3$$
$$\mu_3 (x_3 - 10) = 0$$

The solution is $x_3 = 10$ Mbps and $\mu_1 = 0, \mu_2 = 0, \mu_3 = .055$. The new queueing delay is 

$$q_3 = \frac{0.55 packets/ms}{10 Mbps} \frac{1024 B}{packet} \frac{Mb}{10^6 b} \frac{8 b}{B}  \frac{10^3 ms}{s} (66 ms) = 30 ms$$

and the new window size is

$$w_3 = 117 packets$$

